\newgeometry{top=3.0cm}
\begin{titlepage}
    \begin{center}
        \includegraphics[width=0.15\textwidth]{Report/frontmatter/dtu.pdf}\par\vspace{0.5cm}
    
        {\scshape\LARGE Technical University of Denmark\\}
    	\vspace{0.5cm}
        {\large \coursenumber\;\coursename}\\
    	\vspace{1.5cm}
    	{\huge\bfseries \projectnamelong\\}
    	\vspace{1.5cm}
    	{\Large Steffan Martin Kunoy (s194006)\\}
            {\Large Tjark Petersen (s186083)\\}
        \vspace{1cm}
    	{\Large \thedate\\}
        \vspace{1cm}

    \end{center}    
    \textbf{\textit{Abstract}}\\
        The CRN++ language presented by \citeauthor{soloveichik2018a} is a domain specific language to describe chemical reaction networks (CRN), which provide a basis for performing complex algorithmic computations in chemical solutions.
        
        In this report, we describe the F\# implementation of a parser and interpreter for the CRN++ language, a compiler which translates a CRN to a set of reactions and finally a chemical reaction simulator with a visualization backend which can be used to accurately simulate compiled CRNs. This is based on a thorough analysis of the CRN++ grammar, a definition of a well-formed CRN++ program and associated properties. Our implementation has been used to successfully compile and simulate a selection of example as well as randomly generated programs, with the output of the simulator being validated using the interpreter.
 \vfill
\end{titlepage}
\restoregeometry