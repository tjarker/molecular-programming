\documentclass{article}

% Variables
\newcommand{\coursenumber}{02257}
\newcommand{\coursename}{Applied Functional Programming}
\newcommand{\projectnameshort}{Project 2}
\newcommand{\projectnamelong}{Project 2: \textit{Molecular Programming Language}}
\newcommand{\thedate}{\today}
\renewcommand\thesection{\arabic{section}}

\usepackage{syntax}
\usepackage{graphicx}

\usepackage{biblatex}
\addbibresource{references.bib}

\begin{document}

\newgeometry{top=3.0cm}
\begin{titlepage}
    \begin{center}
        \includegraphics[width=0.15\textwidth]{Report/frontmatter/dtu.pdf}\par\vspace{0.5cm}
    
        {\scshape\LARGE Technical University of Denmark\\}
    	\vspace{0.5cm}
        {\large \coursenumber\;\coursename}\\
    	\vspace{1.5cm}
    	{\huge\bfseries \projectnamelong\\}
    	\vspace{1.5cm}
    	{\Large Steffan Martin Kunoy (s194006)\\}
            {\Large Tjark Petersen (s186083)\\}
        \vspace{1cm}
    	{\Large \thedate\\}
        \vspace{1cm}

    \end{center}    
    \textbf{\textit{Abstract}}\\
        The CRN++ language presented by \citeauthor{soloveichik2018a} is a domain specific language to describe chemical reaction networks (CRN), which provide a basis for performing complex algorithmic computations in chemical solutions.
        
        In this report, we describe the F\# implementation of a parser and interpreter for the CRN++ language, a compiler which translates a CRN to a set of reactions and finally a chemical reaction simulator with a visualization backend which can be used to accurately simulate compiled CRNs. This is based on a thorough analysis of the CRN++ grammar, a definition of a well-formed CRN++ program and associated properties. Our implementation has been used to successfully compile and simulate a selection of example as well as randomly generated programs, with the output of the simulator being validated using the interpreter.
 \vfill
\end{titlepage}
\restoregeometry
\newpage

\tableofcontents

% possible todos:
% opt sim by only calculating concChange for non-catalysts (netchange != 0)

\begin{itemize}
    \item We changed the multiplicity of cmp molecules (such as xgty) from 1 to 2, because deactivated conditional reactions were else still too active
    \item The \texttt{sqrt} module for CRN++ returns the 4th root of the argument. As a result, we modified the interpreter output for the 
    \item Mul(a,b,c) is restricted further by a!=b
    \item Sub(a,b,c) is also restricted by a!=b
\end{itemize}

\section{Introduction}
This report documents the design, implementation and testing of a tool for computing on the basis of chemical reaction networks (CRNs) using the functional programming language F\#. The project is inspired by the formalization of a language for chemical reactions proposed in \textit{CRN++: Molecular Programming Language} \cite{soloveichik2018a}. 

\section{CRN++}

\subsection{Revised Grammar} % Tjark

\subsection{Well-Formed CRN} % Steffan

\subsection{Abstract Syntax Tree Model} % Steffan
% F# type declarations
% Use DrawingTreesLib to draw example CRN++ program

\subsection{Parser for CRN++} % Tjark
% Parsed string property

\subsection{Type Checker for CRN++} % Steffan

\subsection{CRN Generator} % Tjark
% Define custom generator for FsCheck that produces well-formed CRNs

\subsection{CRN States}
% Representation of step outputs from chemical reactions

\subsection{Visualization} % Steffan

\section{Interpreter for CRN++} % Steffan
% used property based testing to verify that order of step commands does not matter

\section{Compiler for CRN++}
% CRN -> list Reaction + initial concentrations
% Chemical clock oscillators
% flag molecules
% helper molecules


\section{Reaction Simulator}
% Simulate chemical reaction by generating sequence of states
% for each Rxn do { for each species in reaction { calc concChange and update changeMap }}
% used PBT to compare interpreter state trace with simulator state trace for random CRNs

\section{Discussion}
% performance


\section{Conclusion}
With this project, we have implemented tools for working with the CRN++ molecular programming language using functional programming in F\#. We have formulated a revised grammar for CRN++ and a set of properties for \textit{well-formed} programs. We also built a parser that takes raw string input and produces CRN++ representations using the defined types in F\#. A type checker is used to evaluate the \textit{well-formed} properties and the default generator was overridden to produce well-formed CRN++ programs. Further, a succinct state representation of chemical reactions was defined, and a visualization component was built to plot state changes over time. 

In the subsequent phases of the project, we worked on interpreting CRN++ programs and compiling these into chemical reactions. Using chemical reactions instead of simple arithmetic for computation raised the complexity level significantly, resulting in some discrepancies from the expected behavior. Nevertheless, all the sample programs computed correctly and were comparable to the interpreted outputs. Overall, the project has provided us with a challenging albeit gratifying experience in functional programming in F\#. 

%We have translated the formal properties of trees into F\# predicates that can be tested using property-based testing with the \texttt{FsCheck} package. Also, we have configured \texttt{FsCheck} with classification of test cases, a custom generator and shrinker as well as automated property tests. Furthermore, using our library allows for visualization of the drawing trees as SVG files, with several customizable options. Overall, the project has provided us a great opportunity to familiarize ourselves with coding in F\# as well as the functional paradigm as a whole.

\newpage
\printbibliography
\end{document}