\section{Conclusion}
With this project, we have implemented tools for working with the CRN++ molecular programming language using functional programming in F\#. We have formulated a revised grammar for CRN++ and a set of properties for \textit{well-formed} programs. We also built a parser that takes raw string input and produces CRN++ representations using the defined types in F\#. A type checker is used to evaluate the \textit{well-formed} properties and the default generator was overridden to produce well-formed CRN++ programs. Further, a succinct state representation of chemical reactions was defined, and a visualization component was built to plot state changes over time. 

In the subsequent phases of the project, we worked on interpreting CRN++ programs and compiling these into chemical reactions. Using chemical reactions instead of simple arithmetic for computation raised the complexity level significantly, resulting in some discrepancies from the expected behavior. Nevertheless, all the sample programs computed correctly and were comparable to the interpreted outputs. Overall, the project has provided us with a challenging albeit gratifying experience in functional programming in F\#. 

%We have translated the formal properties of trees into F\# predicates that can be tested using property-based testing with the \texttt{FsCheck} package. Also, we have configured \texttt{FsCheck} with classification of test cases, a custom generator and shrinker as well as automated property tests. Furthermore, using our library allows for visualization of the drawing trees as SVG files, with several customizable options. Overall, the project has provided us a great opportunity to familiarize ourselves with coding in F\# as well as the functional paradigm as a whole.